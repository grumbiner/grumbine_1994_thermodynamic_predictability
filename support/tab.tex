\documentstyle[12pt]{article}
\textwidth=7.0in
\oddsidemargin=-0.5in
\begin{document}

\begin{center}
Table 1. Magnitude of thermodynamic terms over the ocean and over sea ice.
\linebreak In units of W m$^{-2}$.

\begin{tabular}{ccc}
&&\\
Term & Size - Ice & Size - Water\\
\hline
\hline 

SW $\downarrow$ & 0-350 & 0-350 \\
SW $\uparrow$   & 200   & 35 \\
LW $\uparrow \downarrow $ & 250 & 250 \\
Internal SW & 0-150 & 0-315 \\

& & \\

H & 2-200 & 200 \\
Conduction & 2-300 & 2-30 \\
FW $\uparrow$ & 2-30 & 2-30 \\
LE $\uparrow$ & 2?  & 2? \\
\hline

\end{tabular}
\end{center}

\pagebreak
\begin{center}
Table 2.  Albedo representation for different sea ice and snow states
\linebreak 
(Shine and Henderson-Sellers, 1985).

\begin{tabular}{llll}
&&&\\
Albedo Class    & Symbol &    Value\\
\hline
\hline

Dry Snow        &   $ \alpha_d  $ &    0.80\\
Thick Melting Snow& $ \alpha_m  $ &    0.65\\
Thin Melting Snow&  $\alpha_{mb}$ &  $\alpha_b + (h_s/0.10)
                                   (\alpha_m-\alpha_b)$ \\
Bare Puddled Ice&   $\alpha_b  $   &    0.53 \\
Bare Frozen Ice &   $\alpha_{bf}$  &    0.72 \\

Thin Forming Ice &  $\alpha_{btf}$ &  
 $\alpha_{btm}$ & $0.0 \leq h_i \leq 1.0$ \\
& & $0.472 + 2.0(\alpha_{bf} - 0.472)(h_i -1.0)$ 
& $1.0 \leq h_i \leq 1.5$ 
\\

Thin Melting Ice &  $\alpha_{btm}$ &
$0.472 + 2.0(\alpha_b-0.472)(h_i-1.0)$ & 
    $ 1.0 \leq h_i \leq 1.5 $ \\
& & $0.2467 + 0.7049 h_i - 0.8608 h_i^2 + 0.3812 h_i^3$ &
    $ 0.05 \leq h_i \leq 1.0$  \\
& & $ 0.1 + 3.6 h_i$ & $h_i \leq 0.05$ \\

Thin Snow On Frozen Ice&  $ \alpha_{df}$ &
$\alpha_d$ & $h_i \geq 0.05$ \\
& & $\alpha_{btf} + (h_s / 0.05) (0.8 - \alpha_{btf} )$ & $h_s \leq 0.05$ ;
    $ h_i \leq 1.5$ \\
& & $\alpha_{bf} + (h_s / 0.05) (0.08 - \alpha_{bf}$ &$ h_s \leq 0.05 $ ;
    $ h_i \geq 1.5$ \\
\hline
\end{tabular}
\end{center}

\pagebreak
\begin{center}
Table 3.  Precision required in forcing terms for desired accuracy in
forecast ice motion.
\begin{tabular}{ccccccc}
&&&&&&\\
                 & Drift& Rule&      & & Force  \\
Precision (1 wk) & U$_A$ & P (Pa) &
      H$_{topo}$ (cm) & f & U$_{A}$ (5 m/s) & U$_{O}$ (10 cm/s) \\
\hline
\hline
1 km & 0.2 & 3 & 0.22 & 2\% & $\pm$0.02 & $\pm$ 0.08 \\
25 km & 5 & 80 & 5.3 & 40\% & $\pm$0.40 & $\pm$ 2.3 \\
Operational & 2? & 200? & 10? & --- & 2? & not \\
\hline
\end{tabular}
\end{center}

\pagebreak
Table 4.  Required accuracy in thermodynamic fluxes to predict the 
growth of thin (10 cm) and thick ice (1 m) in W m$^{-2}$, and the
relative magnitude of this flux compared with the size of individual
elements in the total flux. $\alpha$ is the albedo, LW is the longwave
flux, FW is the ocean-supplied heat flux, K$_{thin}$ is the thermal
conduction through thin ice, K$_{thick}$ is the thermal conduction 
through thick ice, and S$_{thin, thick}$ is the thermal conduction 
through thin or
thick ice with a 10 cm layer of snow.
\begin{center}
\begin{tabular}{llllllllll}
&&&&&&&&&\\
Time & Thin & Thick & $\alpha_{thin}$ & 
     LW$_{Thin}$ &  FW$_{Thin}$ & K$_{Thin}$ & K$_{Thick}$  
                                & S$_{Thin}$ & S$_{Thick}$\\
\hline
\hline
Week & 50 & 500 & 0.14 & 0.25 &  --- &  0.125 & --- & 1.0 & --- \\
Month & 11 & 110 & 0.032 &0.05 &  1.0 & 0.025 & --- & 0.2 & ---\\
Year & 1 & 10 & 0.003 & 0.005 &  0.1 &  0.0025 & 0.25 & 0.02 & 0.4 \\
\hline
Observation &&&&&&&&\\
Model &&&&&&&&\\
\hline
\end{tabular}
\end{center}

\pagebreak
\begin{center}
Table 5.  Research and operations time lines.

\underline{Time Line - Research}
\begin{tabular}{cll}
Year & Scientist & Physics \\
\hline
\hline
1902 & Nansen & Drift Rule - Arctic \\
1945 & Zubov & Drift Rule - Arctic \\
&& Empirical Ice Thickness \\
1965 & Skiles & Drift Rule - Arctic \\
1965 & Campbell & Viscous Ice Pack Rheology \\
1971 & Maykut and Untersteiner & Detailed Floe Thermodynamics \\
1975 & Thorndike et al. & Ice Thickness Distribution Theory \\
1976 & Semtner & Simplified Floe Thermodynamics \\
1979 & Hibler & Viscous-Plastic Ice Rheology\\
1980 & Hibler & Ice Thickness Distribution Application \\
1982 & Thorndike and Colony & Drift Rule - Arctic \\
1983 & Smith & Ice Rheology Reconsidered \\
&& --- Hibler 1979 is nonphysical\\
1984 & Hibler and Bryan & Coupled Ice-Ocean with 1980 Hibler Ice \\
1987 & Semtner & Coupled Ice-Ocean with simplified rheology \\
1988 & Overland and Pease & Local Scale ice rheology\\ 
&& --- derived following Smith-like guidelines\\
&& Coupled to ocean \\
1990 & Martinson and Wamser & Drift Rule - Antarctic \\
\hline
&&\\
&&\\
\end{tabular}


\underline{Time Line - Operational Objective}
\begin{tabular}{cclc}
Period & Site & Physics& Year \\
\hline
\hline
?-Present & NMC & Skiles Drift Rule & 1965\\
1965-1983 & FNOC & Skiles Drift Rule & 1965 \\
1975-?    & FNOC & Zubov Thermodynamics & 1945 \\
1983-1987 & FNOC & Thorndike and Colony Drift Rule & 1982 \\
1987-present & FNOC & Hibler Rheology & 1980\\
&&                    Semtner Thermodynamics & 1976\\
10/92 Schedule & FNOC & Hibler-Bryan Coupled Ice-Ocean & 1984 \\
\hline

\end{tabular}
\end{center}

\end{document}
